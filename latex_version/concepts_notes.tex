% LATEX TEMPLATE 
% adapted https://infinitedescent.xyz/latex/ by wade
% other minor adjustments cited below

\documentclass[11pt]{article}

% Edit the following to change the title, author name and date
\title{\LaTeX{} Condensed Concepts Notes}
\author{saffron\_}
\date{}

% Packages
\usepackage{amsmath}
\usepackage{amsfonts}
\usepackage{amssymb}
\usepackage{amsthm}
\usepackage{enumerate}
\usepackage{geometry}
\usepackage{graphicx}
\usepackage{hyperref}

% code block shenanigans
\usepackage{listings}
\usepackage{color}

\definecolor{dkgreen}{rgb}{0,0.6,0}
\definecolor{gray}{rgb}{0.5,0.5,0.5}
\definecolor{mauve}{rgb}{0.58,0,0.82}

\lstset{frame=tb,
  language=C,
  aboveskip=3mm,
  belowskip=3mm,
  showstringspaces=true,
  columns=fixed,
  basicstyle={\small\ttfamily},
  numbers=left, % none, left
  numberstyle=\tiny\color{gray},
  numbersep=5pt,
  keywordstyle=\color{blue},
  commentstyle=\color{dkgreen},
  stringstyle=\color{mauve},
  breaklines=true,
  breakatwhitespace=true,
  tabsize=4
}

% Page setup
\setlength{\parskip}{10pt}
\setlength{\parindent}{0pt}
\geometry{
    paper={letterpaper}, % Change to 'a4paper' for A4 size
    marginratio={1:1},
    margin={1.25in}
}

% Theorem environments
% also see https://tex.stackexchange.com/questions/53978/custom-theorem-numbering
\theoremstyle{definition}
\newtheorem{theorem}{Theorem}
\newtheorem{lemma}[theorem]{Lemma}
\newtheorem{corollary}[theorem]{Corollary}
\newtheorem{proposition}[theorem]{Proposition}
\newtheorem{definition}[theorem]{Defn}
\newtheorem{example}[theorem]{Example}

% custom for this document
\newcommand{\addsection}[1]{\addcontentsline{toc}{section}{#1}\section*{#1}}
\newcommand{\bb}[1]{\mathbb{#1}} 
\newcommand{\floor}[1]{\left\lfloor #1 \right\rfloor}
\newcommand{\ceil}[1]{\left\lceil #1 \right\rceil}
\DeclareMathOperator{\lcm}{lcm}
\DeclareMathOperator{\Img}{Im}
\DeclareMathOperator{\PreIm}{PreIm}

\begin{document}
\maketitle
\tableofcontents
\newpage

%%%%%%%%%%%%%%%%%%%%%%%%%%%%
%% Start of document body %%
%%%%%%%%%%%%%%%%%%%%%%%%%%%%

\newpage
\addsection{Number Sets}
\begin{itemize}
    \item set: collection of objects
    \item $x \in X$ means x is an element of X (extends to $\notin$)
    \item $\bb{N}$: natural numbers ($\{0,1,2,3,\dots\}$)
    \begin{itemize}
        \item domain $[0, \infty)$
        \item $+ / \times$ of two $\bb{N}$ is still $\bb{N}$
        \item has commutativity, associativity, distributivity of $+ / \times$
    \end{itemize}
    \item $\bb{Z}$: integers ($\{\dots ,-3,-2,-1,0,1,2,3, \dots\}$)
    \begin{itemize}
        \item domain $(-\infty,\infty)$
        \item $+/-/\times$ of two $\bb{Z}$ is still $\bb{Z}$
        \item $\bb{Z}^+$: positive integers ($\{1,2,3,4, \dots\}$)
        \item \textbf{Defn:} For $a,b\in\bb{Z},$ we say 
        $a\ divides\ b\ (a\mid b),$ iff $\exists c \in \bb{Z}$ s.t. $b=ac$
        \item \textbf{Thrm 2.1.1} (Divisibility is Transitive).
        \emph{Let $a,b,c \in \bb{Z}.$ If $c|b$ and $b|a$, then $c|a$}
        \item \textbf{Defn:} Let $n \in \bb{Z}$. 
        Then $n$ is even iff $2\mid n$ and odd iff $2\nmid n$
        \item \textbf{Thrm 2.1.2} (The Division Theorem).
        \emph{Let $a,b \in \bb{Z}$ with $b \neq 0$.
        There are unique integers $q,r$ s.t. $a=bq+r$ with $0\le r< |b|$} \\
        ie $\forall a,b\in\bb{Z},(b\ne 0\Rightarrow \exists ! q,r\in\bb{Z},(a=bq+r \land 0\le r < |b|))$
    \end{itemize}
    \item $\bb{Q}$: rational numbers (all numbers $\frac{a}{b}$
    where $a,b \in \bb{Z}$ and $b \neq 0$)
    \begin{itemize}
        \item domain $(-\infty,\infty)$
        \item $+/-/\times/\div (\text{except 0})$ of two $\bb{Q}$ is still $\bb{Q}$
    \end{itemize}
    \item $\bb{R}$: real numbers (the entire number line)
    \begin{itemize}
        \item domain $(-\infty,\infty)$
        \item $+/-/\times/\div (\text{except 0})$ of two $\bb{R}$ is still $\bb{R}$
        \item $\times$ two nonzero $\bb{R}$ is nonzero (applies to the rest)
        \item real $r\in \bb{R}$ is \emph{irrational} iff $r\in\bb{Q}$
    \end{itemize}
\end{itemize}

\newpage
\addsection{Sets}
\subsection*{Polynomials over Number Sets}
\begin{itemize}
    \item \emph{single variable polynomial} over $S$ with respect to $x$:
    \[ a_nx^n + a_{n-1}x^{n-1} + \dots + a_1x + a_0 
    \text{ where } n\in\bb{N} \text{ and each } a_i\in S \text{ for } 0\le i \le n \]
    \begin{itemize}
        \item $a_0,a_1,\dots,a_n$ are called \emph{coefficients}
        \item \emph{degree} of a polynomial: largest $i\in\bb{N}$ s.t. $a_i\ne0$
        \item \emph{zero polynomial} defined as degree $-\infty$
        \item denoted $p(x)$ where $x$ is the indeterminate
    \end{itemize}
    \item $S[x]$, read ``$S$ adjoin $x$":
    set of all polynomials with coefficients from $S$
    \begin{itemize}
        \item \textbf{eg:} $\bb{Z}[x]$ is set of poly.s with integer coeff.s --- 
        $5x^2 +2x \in \bb{Z}[x], \in \bb{Q}[x], \text{ and }\in \bb{R}[x]$
    \end{itemize}
\end{itemize}

\subsection*{Set Notation}
\begin{itemize}
    \item \textbf{Roster notation} (informal): $A = \{1,2\}$, $B = \{2,4,6,\dots\}$
    \item \textbf{Set-Builder notation:} 
    $\{x\in S\mid p(x)\}, A=\{x\in\bb{N} \mid x\le 5\}$ (use $\mid$ or :)
    \item Alternate Set-Builder (informal): 
    $\{\emph{expression}(x) \mid x\in S\}, E=\{ 2n \mid n\in\bb{Z}\}$
\end{itemize}

\subsection*{Other Special Sets}
\begin{itemize}
    \item $\varnothing$ = empty set = \{\}
    \item for $n\in\bb{N}, [n] = {\text{first n positive }\bb{Z}}$ ($[3] = \{1,2,3\}$)
\end{itemize}

\subsection*{Set Equality and Subsets}
\begin{itemize}
    \item set A is a \emph{subset} of set B ($A\subseteq B$) iff 
    every element in A is also an element of B
    \item $A=B$ iff $A\subseteq B$ and $B\subseteq A$
    \item $A \not\subseteq B$: A not a subset of B
    \item $A \subsetneq B$: A is a \emph{proper subset} of B (ie $A\subseteq B, A\ne B$)
\end{itemize}

\newpage
\addsection{Logic}
\subsection*{Symbolic Notation}
\begin{itemize}
    \item For $x\in\bb{R}$,
    \begin{itemize}
        \item \emph{floor} of $x$: greatest $n\in\bb{Z}, n\le x$.
        \item \emph{ceiling} of $x$: least $n\in\bb{Z}, x\le n$.
    \end{itemize}
    \item \emph{propositional variable:} a symbol representing the proposition (eg $p,q,$ etc)
    \item \emph{propositional formula:} either a prop. variable or expression built from them and connectives (logical operators)
    \begin{itemize}
        \item \textbf{Conjunction} ($p\wedge q$): ``and''
        \item \textbf{Disjunction} ($p\vee q$): ``or''
        \item \textbf{Negation} ($\neg p$): ``not''
        \item \textbf{Logical Implication} ($p \Rightarrow q$): ``if $p$ then $q$''
        \begin{itemize}
            \item $p$ is called the \emph{hypothesis}, \emph{supposition}, or \emph{antecedent}
            \item $q$ is called the \emph{consequent} or \emph{conclusion}
            \item $q \Rightarrow p$ is the \emph{converse} of $p \Rightarrow q$
            \item \begin{tabular}{c|c|c}
                $p$ & $q$ & $p \Rightarrow q$\\ \hline
                T & T & T \\
                T & F & F \\
                F & T & T \\
                F & F & T
            \end{tabular}
        \end{itemize}
        \item \textbf{Biconditional Operator} ($p \Leftrightarrow q$): ``p iff q''
        \begin{itemize}
            \item if $p \Leftrightarrow q$ is a tautology, $p$ and $q$ are \emph{logically equivalent} ($\equiv$)
            \item \begin{tabular}{c|c|c|c|c}
                $p$ & $q$ & $p \Rightarrow q$ & $q \Rightarrow p$ & $p \Rightarrow q \wedge q \Rightarrow p$ aka $p \Leftrightarrow q$\\ \hline
                T & T & T & T & T \\
                T & F & F & T & F \\
                F & T & T & F & F \\
                F & F & T & T & T
            \end{tabular}
        \end{itemize}
    \end{itemize}
    \item \emph{tautology:} propositional formula that is always \textbf{true} no matter how T/F is assigned
    \item \emph{contradiction:} proposition that is known or assumed to be \textbf{false}
\end{itemize}    
\newpage
\subsection*{Equivalences}
\begin{itemize}
    \item \textbf{Thrm 3.2.1} (DeMorgan's Laws for Connectives).
    \begin{enumerate}[(i)]
        \item $\neg(p\wedge q) \equiv \neg p \vee \neg q$
        \item $\neg(p\vee q) \equiv \neg p \wedge \neg q$
    \end{enumerate}
    \item Other important equivalences
    \begin{itemize}
        \item $p \wedge (q\vee r) \equiv (p\wedge q) \vee (p\wedge r)$ (distributivity)
        \item $p \vee (q \wedge r) \equiv (p \vee q) \wedge (p \vee r)$ (distributivity part 2)
        \item (commutativity, associativity, double negation)
        \item $p \Leftrightarrow q \equiv (p \Rightarrow q) \wedge (q \Rightarrow p)$ (biconditional equivalence)
        \item $p \Rightarrow q \equiv \neg q \Rightarrow \neg p$ (contraposition)
        \item $p \Rightarrow q \equiv \neg p \vee q$ (disjunctive form of implication)
    \end{itemize}
\end{itemize}
\subsection*{Quantifiers}
\begin{itemize}
    \item a variable $x$ is \emph{free} iff you can sub in elements for $x$. Otherwise, it is \emph{bound}.
    \item \emph{predicate:} statement involving free variables, denoted $p(x_1,x_2,\dots,x_n)$
    \item \emph{quantifiers:} $\forall$: ``for all," $\exists$: ``there exists," $\exists!$: ``there exists only one"
\end{itemize}
\subsection*{Maximally Negating Propositions}
\begin{itemize}
    \item \textbf{Thrm 5.1.1} (DeMorgan's Laws for Quantifiers).
    \begin{enumerate}[(i)]
        \item $\neg (\forall x\in S, p(x)) \equiv \exists x\in S, \neg p(x)$
        \item $\neg (\exists x\in S, p(x)) \equiv \forall x\in S, \neg p(x)$
    \end{enumerate}
    \item a proposition is \emph{maximally negated} iff only $\neg$ s appear immediately before a predicate or propositional variable
\end{itemize}

\newpage
\addsection{Proof Writing}
\begin{itemize}
    \item Proving Universal Statements ($\forall x\in S, p(x)$)
    \begin{itemize}
        \item Direct: let $x \in S$ be arbitrary and fixed, prove p(x) true
        \item Indirect: AFSOC $\exists x\in S, \neg p(x)$ and find a contradiction
    \end{itemize}
    \item Proving Existential Statements ($\exists x\in S, p(x)$)
    \begin{itemize}
        \item Direct: give an element $x\in S$, show $p(x)$ is true
        \item Indirect: AFSOC $\forall x\in S, \neg p(x)$ and find a contradiction
    \end{itemize}
    \item Proving Conditional Statements ($p \Rightarrow q$)
    \begin{itemize}
        \item Direct: assume $p$ is true. show $q$ is true
        \item Indirect:
        \begin{itemize}
            \item by contraposition: recall $\neg q \Rightarrow \neg p$. assume $\neg q$. show $\neg p$.
            \item by contradiction: assume $p \wedge \neg q$ and find a contradiction
        \end{itemize}
    \end{itemize}
    \item Proving Biconditional Statements ($p \Leftrightarrow q$)
    \begin{itemize}
        \item prove $p \Rightarrow q$ and $q \Rightarrow p$ 
    \end{itemize}
    \item Proving Disjunctions ($\vee, \wedge$)
    \begin{itemize}
        \item proving $(p \vee q) \Rightarrow r$ directly (with 2 cases)
        \begin{itemize}
            \item \textbf{Case 1:} assume $p$ holds, show $r$ holds
            \item \textbf{Case 2:} assume $q$ holds, show $r$ holds
            \item no distinction between cases 1 and 2? use WLOG 
        \end{itemize}
        \item proving $p \Rightarrow (q \vee r)$ directly
        \begin{itemize}
            \item assume $p$ is true
            \item if $q$ holds then we're done, so...
            \item assume $\neg q$ holds and prove $r$ holds
        \end{itemize}
    \end{itemize}
    \item Proving Existence and Uniqueness ($\exists!x\in S,p(x)$)
    \begin{itemize}
        \item Existence: prove $\exists x \in S, p(x)$
        \item Uniqueness: Let $a,b \in S$ s.t. both $p(a)$ and $p(b)$ hold. show that $a=b$
    \end{itemize}
\end{itemize}

\newpage
\addsection{Sets (Part 2)}
\begin{itemize}
    \item \emph{power set of $A$} ($\mathcal{P}(A)$): set of all subsets of $A$
    \begin{itemize}
        \item for any set $A$, $\varnothing \in \mathcal{P}(A)$ and $A\in\mathcal{P}(A)$
    \end{itemize}
\end{itemize}
\subsection*{Set Proofs}
\begin{itemize}
    \item \emph{Containment} (prove $A \subseteq B$): Fix an arbitrary $a\in A$, show $a\in B$, conclude $a\subseteq B$
    \item \emph{Double Con.} (prove $A=B$): prove $A\subseteq B$, prove $B\subseteq A$, conclude $A=B$
    \item chain of $\Leftrightarrow$'s (prove $A=B$): fix arb $x\in U$, show $x\in A \Leftrightarrow x\in B$ with iff's
\end{itemize}
\subsection*{Set Operations}
\begin{itemize}
    \item \textbf{Set Intersection}: $A\cap B = \{ x\in U \mid x\in A\land x\in B \}$ ($A$ and $B$ are \emph{disjoint} iff $A\cap B = \varnothing$)
    \item \textbf{Set Union}: $A\cup B = \{ x\in U \mid x\in A \lor x\in B\}$
    \item \textbf{Set Difference}: $A\setminus B = \{ x\in U\mid x\in A \land x\not\in B\}$
    \item \textbf{Family of Sets} indexed by I: sets $A_i$ for each $i\in I$, denoted $\{A_i\mid i\in I\}$ or $\{A_i\}_{i\in I}$
    \item \textbf{Indexed Intersection}: $\displaystyle \bigcap_{i\in I}A_i = \{x\in U \mid \forall i\in I, x\in A_i\}$
    \item \textbf{Indexed Union}: $\displaystyle \bigcup_{i\in I}A_i = \{x\in U \mid \exists i\in I, x\in A_i\}$
    \item \textbf{Cartesian Product} of A and B: $A\times B = \{(a,b)\mid a\in A, b\in B\}$
    \begin{itemize}
        \item notation: $\bb{R}\times\bb{R} = \bb{R}^2$
        \item $A\times \varnothing = \varnothing \times A = \varnothing$
        \item $A_1\times A_2 \times\dots\times A_n = \{(a_1, a_2,\dots,a_n)\mid a_1\in A_1 \land a_2\in A_2 \land \dots\land a_n\in A_n\}$\\
        $\displaystyle =\prod_{i=1}^{n}A_i$ $(a_1,a_2,\dots,a_n)$: ordered n-tuple (eg $\bb{R}\times\bb{R}\times\bb{R} = \bb{R}^3$)
    \end{itemize}
\end{itemize}
\newpage
\subsection*{Set Properties}
\begin{itemize}
    \item \textbf{Thrm 9.1.1} (Properties of Unions and Intersections). \emph{letting $A$, $B$ be sets,}
    \begin{itemize}
        \item $A\cap B = B\cap A$
        \item $A\cup B = B\cup A$
        \item $A\cap B \subseteq A$
        \item $A \subseteq A \cup B$
        \item $A\subseteq B \Leftrightarrow A\cap B = A$
    \end{itemize}
    \item \textbf{Thrm 9.1.2} (DeMorgan's Laws for Sets). \emph{For any sets A, X, and Y, and if \\$\{X_i \mid i \in I\}$ is an indexed family of sets,}
    \begin{itemize}
        \item $A\setminus (X\cup Y) = (A\setminus X)\cap(A\setminus Y)$
        \item $A\setminus (X\cap Y) = (A\setminus X)\cup(A\setminus Y)$
        \item $\displaystyle A\setminus\bigcup_{i\in I}X_i = \bigcap_{i\in I}(A\setminus X_i)$
        \item $\displaystyle A\setminus\bigcap_{i\in I}X_i = \bigcup_{i\in I}(A\setminus X_i)$
    \end{itemize}
\end{itemize}
good luck on exam 1 $<$3

\newpage
\addsection{Functions}
\begin{itemize}
    \item a \emph{function} from a domain set $X$ to a codomain set $Y$ is a specification of elements $f(x)\in Y$ for each $x\in X$ s.t. $\forall x\in X, \exists ! y\in Y, y= f(x)$
    \item given a mapping $f : A \rightarrow B$, $f$ is a function iff$\dots$
    \begin{itemize}
        \item $\forall a\in A, f(a)$ is defined \textbf{i.e.} domain = $A$
        \item $\forall a\in A, f(a) \in B$ \textbf{i.e.} codomain = $B$
        \item $\forall a, a' \in A, (a=a' \Rightarrow f(a)=f(a'))$ \textbf{i.e.} uniqueness, vertical line test
    \end{itemize}
    \item To define the following, let $f:A\rightarrow B$ be a function.
    \item $f=g$ iff $\forall a\in A, f(a) = g(a)$ \textbf{i.e.} for all inputs you get the same outputs
    \item the \emph{graph} of $f$, $Gr(f) = \{(a,b)\in A\times B\mid b = f(a)\} \subseteq A\times B$
    \item for $X\subseteq A$, \emph{image of $X$ under $f$}: $\Img_f(X) = f[X] = \{b\in B\mid \exists x\in X, f(x) = b\}$ 
    \begin{itemize}
        \item \emph{image of $f$} (of the entire domain): $\Img(f) = \{b\in B\mid\exists a\in A,f(a)=b\}$
    \end{itemize}
    \item for $Y\subseteq B,$ \emph{preimage of $Y$ under $f$}: $\PreIm_f(Y) = f^{-1}[y] = \{a\in A\mid f(a)\in Y\}$
    \item $f$ is \emph{injective/1-to-1} iff $\forall x,y\in A,(f(x) = f(y) \Rightarrow x=y)$ (ie $f$ passes HLT)
    \item $f$ is \emph{surjective/onto} iff $\forall b\in B, \exists a\in A, f(a)=b$ (ie $\Img(f)=B$)
    \item $f$ is a \emph{bijection} iff $f$ is both a injection and a surjection
    \item $h$ is a \emph{composition} of $g$ with $f$, denoted $h= g\circ f$ iff $h(a) = g(f(a))$
    \begin{itemize}
        \item when $A,B,C$ are sets and $f: A\rightarrow B, g:B\rightarrow C$ are functions. creates $h:A\rightarrow C$
        \item as long as $\Img(f) \subseteq domain(g)$, this operation works
        \item \textbf{Thrm 14.1.1} (Associativity of Comp.). Let $f:A\rightarrow B, g:B\rightarrow C, h:C\rightarrow D$.
        \begin{itemize}
            \item $h\circ (g\circ f) = (h \circ g) \circ f$
        \end{itemize}
    \end{itemize}
    \item bruh where was the identity function
    \item \textbf{Thrm 14.1.2}. Let $f:A\rightarrow B, g:B\rightarrow C$.
    \begin{itemize}
        \item If $f$ and $g$ are in/sur/bijections, then $g \circ f$ is an in/sur/bijection.
    \end{itemize}
    \item Let $f: A\rightarrow B, g:B\rightarrow C$.
    \begin{itemize}
        \item $g$ is a \emph{left inverse} for $f$ iff $g\circ f = id_A$ ($f$ is injective)
        \item $g$ is a \emph{right inverse} for $f$ iff $f \circ g = id_B$ ($f$ is surjective)
        \item $g$ is a \emph{(2-sided) inverse} for $f$ iff g is a left and right inverse
        \item $f$ is \emph{invertible} iff f has a (2-sided) inverse
        \item \textbf{Thrm 14.1.3} (Uniqueness of Inverses). If $f$ is invertible than its inverse $f^{-1}$ is unique.
        \item \textbf{Thrm 14.1.4}. $f$ is invertible iff $f$ is a bijection. (ie can prove $f:A\rightarrow B$ is a bijection by making a well defined $g:B\rightarrow A$ and proving $g$ is an inverse of $f$)
    \end{itemize}
\end{itemize}
\subsection*{wtf the natural numbers 2 electric boogaloo???}
TODO
\begin{itemize}
    \item informal definition
    \item formal definition w/ peano's axioms for $\bb{N}$
    \begin{itemize}
        \item $0 \not\in \Img_S(\bb{N})$
        \item $s$ is injective
        \item For all sets $X$, if $0\in X$ and $\forall n\in N, (n\in X\rightarrow s(n)\in X)$ then $\bb{N}\subseteq X$.
    \end{itemize}
    \item \textbf{Thrm 15.2.1} (Recursion Theorem).
    \item math with $\bb{N}$ 2, electric boogaloo
    \item recursive function definitions
    \[ 
        \sum\limits_{k=1}^{n}a_k = \left\{
            \begin{array}{ll}
                \sum\limits_{k=1}^{0}a_k = 0\\
                \\
                \sum\limits_{k=1}^{m+1}a_k = (\sum\limits_{k=1}^{m}a_k) + a_{m + 1} \text{  if } n=m+1\\
            \end{array} 
        \right.
    \]
    \[ 
        \prod\limits_{k=1}^{n}a_k= \left\{
            \begin{array}{ll}
                \prod\limits_{k=1}^{0}a_k = 1\\
                \\
                \prod\limits_{k=1}^{m+1}a_k = (\prod\limits_{k=1}^{m}a_k) \cdot a_{m + 1} \text{  if } n=m+1\\
            \end{array} 
        \right. 
    \]
    \[ n! = \sum\limits_{k=1}^{n}k \]
\end{itemize}

\newpage
\addsection{The Principle of Mathmatical Induction (PMI)}
\begin{itemize}
    \item \textbf{Thrm 16.2.1} (PMI).
    \begin{itemize}
        \item WTS $p(n)$ is true for all natural numbers $n\ge n_0$.
        \item Let $p(n)$ be a predicate defined on $\bb{N}, n_0\in\bb{N},$ and $S=\{n\in\bb{N} \mid n\ge n_0\}$.
        \item To prove $\forall n\in S,p(n) \dots$
        \begin{itemize}
            \item (Base Case) Verify that $p(n_0)$ holds.
            \item (Inductive Step) Fix $n\in S$ and assume $p(n)$ holds. 
            \begin{itemize}
                \item \textbf{or for strong pmi:} Fix that $n\in S$ s.t. $\forall i\in S,$ with $n_0\le i\le n, p(i)$ holds. $n_0$ should be the last base case if there are multiple.
            \end{itemize} 
            \item Prove that $p(n+1)$ must also be true. 
            \item By PMI, we conclude $\forall n\in S,p(n)$. $\qed$
        \end{itemize}
    \end{itemize}
\end{itemize}
\addsection{Well-Ordering Principle}
\begin{itemize}
    \item \emph{well-ordered:} a set of which every nonempty subset has a least element
    \item \textbf{Thrm 19.1.1} (Well-Ordering-Principle). $\bb{N}$ is a well-ordered set.
    \item Proof by Infinite Descent ($\forall n\in \bb{N}, p(n)$)
    \begin{itemize}
        \item AFSOC $\exists n\in \bb{N}$ st $\neg p(n)$ holds.
        \begin{itemize}
            \item ie $S = \{n\in\bb{N}\mid \neg p(n)\}\ne \varnothing$
        \end{itemize}
        \item Let $n\in S$ be the least such element, by the WOP.
        \item Show $\exists k\in S$ with $k<n$ $\rightarrow\leftarrow$ contradicts the minimality of $n$.
        \item Conclude that $\forall n\in\bb{N}, p(n)$. 
    \end{itemize}
\end{itemize}

\newpage
\addsection{Binary Relations}
\begin{itemize}
    \item A binary relation links or compares two elements.
    \item A \emph{binary relation} from $S$ to $T$ is a predicate $R(s,t)$ defined on $S\times T$.
    \begin{itemize}
        \item $S$ is the \emph{domain} of $R$
        \item $T$ is the \emph{codomain} of $R$
        \item If $S=T$ then $R$ is a \emph{homogenous relation} and we say ``$R$ is a relation on $S$''
        \item can also be written as the \emph{graph} of R, Gr($R$) = $\{(s,t)\in S\times T\mid R(s,t)\}$
        \item \textbf{Thrm 19.2.1}. Let $S$ and $T$ be sets. Every subset of $S\times T$ is the graph of a unique relation from $S$ to $T$.
    \end{itemize}
    \item the \emph{discrete relation} from $S$ to $T$
    \begin{itemize}
        \item Gr$(R)=S\times T$
        \item ie $\forall s\in S, \forall t\in T, R(s,t)$
        \item everything is related to everything
    \end{itemize}
    \item the \emph{empty relation} from $S$ to $T$
    \begin{itemize}
        \item Gr$(R) = \varnothing$
        \item ie $\forall s\in S, \forall \in T, \neg R(s,t)$
        \item nothing is related to anything
    \end{itemize}
    \item \textbf{Defn:} (Congruence Modulo $m$). Let $m\in \bb{Z}^+$ and $a,b \in \bb{Z}$. $a$ is congruent to $b$ modulo $m$, denoted $a\equiv b$ (mod $m$) or $a\equiv_m b$, iff $m \mid a-b$.
    \begin{itemize}
        \item \textbf{Thrm 20.1.1}. Congruence modulo m is an equivalence relation.
        \item For $m\in \bb{Z}^+$, the set of equivalence classes for $\equiv_m$, called \emph{congruence classes}, is denoted $\mathbb{Z}/m\mathbb{Z}$.
        \begin{itemize}
            \item eg, $\bb{Z}/3\bb{Z} = \{[0]_3,[1]_3,[2]_3\}$
        \end{itemize}
    \end{itemize}
    \item \textbf{Defn:} Let $R$ be an equivalence relation on a set $S$.
    \begin{itemize}
        \item For $x\in S$ we define the \emph{equivalence class of $x$ under $R$}, $[x]_R=\{y\in S\mid R(x,y)\}$ = the set of elements in $S$ which are equivalent to $x$.
        \item the set of all equivalence classes is called \emph{$S$ modulo $R$}, $S/R=\{[x]_R\mid x\in S\}$
    \end{itemize}
\end{itemize}
\newpage
\subsection*{Properties of Homogenous Relations}
Let $R$ be a relation on $S$. Then $R$ is called
\begin{itemize}
    \item \emph{reflexive} iff $\forall x\in S, R(x,x)$
    \item \emph{irreflexive} iff $\forall x\in S, \neg R(x,x)$
    \item \emph{symmetric} iff $\forall x,y \in S, (R(x,y)\rightarrow R(y,x))$
    \item \emph{antisymmetric} iff $\forall x,y\in S, (R(x,y) \land R(y,x) \rightarrow x=y)$ (or equivly, its contrapos.)
    \item \emph{transitive} iff $\forall x,y,z\in S, ((R(x,y)\land R(y,z)) \rightarrow R(x,z))$
    \item \emph{total} iff $\forall x,y\in S, (x\ne y\rightarrow (R(x,y)\lor R(y,x)))$
    \item \emph{an equivalence relation} iff $R$ is reflexive, symmetric, and transitive
\end{itemize}
\subsection*{Partitions}
\begin{itemize}
    \item \textbf{Defn:} Let $S$ be a set, $I$ an index set, and $A_i\in\mathcal{P}(S)$ for each $i\in I$. The indexed family of sets $\{A_i\mid i\in I\}$ is a \emph{partition} of $S$ iff
    \begin{enumerate}
        \item $\forall i\in I, A_i \ne \varnothing$
        \item $\forall i,j\in I, (A_i = A_j \lor A_i \cap A_j = \varnothing)$
        \item $\bigcup_{i\in I}A_i = S$
    \end{enumerate}
    \item \textbf{Thrm} (Fundamental Thrm of Equivalence Relations): Let $S$ be a nonempty set.
    \begin{enumerate}
        \item If $R$ is an equivalence relation on $S$, then $S/R$ is a partition of $S$.
        \item If $\mathcal{F}$ is a partition of $S$ then there exists an equivalence relation $R$ on $S$ s.t. $S/R=\mathcal{F}$
    \end{enumerate}
\end{itemize}
\subsection*{Order Relations}
Let $R$ be a binary relation on a set $S$.
\begin{itemize}
    \item $R$ is a \emph{partial order of} $S$ iff $R$ is reflexive, antisymmetric, and transitive.
    \item If $R$ is a partial order on $S$, then we call $(S,R)$ a \emph{poset} (eg $(\mathcal{P}(x),\subseteq)$ and $(\bb{R}, \le)$)
    \item $R$ is a \emph{strict partial order} of $S$ iff $R$ is irreflexive, antisymmetric, and transitive. (eg $(\mathcal{P}(x),\subsetneq)$ and $(\bb{R},<)$)
    \item $R$ is a \emph{total order} or \emph{linear order} iff $R$ is a partial order and also total
\end{itemize}
good luck on exam 2 $<$3

\newpage
\addsection{Cardinality}
\begin{itemize}
    \item \textbf{Defn:} Two sets $A$ and $B$ have the same \emph{cardinality} (aka are \emph{equinumerous}), iff there exists a bijection $f:A\rightarrow B$.
    \item A set $X$ is \emph{finite} iff $\exists n\in\bb{N}$ and a bijection $f:[n]\rightarrow X$. We denote this $|X|=n$.
    \item A set $X$ is \emph{infinite} iff $X$ is not finite.
    \item \textbf{Thrm:} If $X$ is a finite set then $\exists! n\in\bb{N}, |X|=n$. This has some corollaries.
    \begin{itemize}
        \item For any $n\in\bb{N}$, all subsets of $[n]$ are finite.
        \item If $f:A\rightarrow B$ is an injection and $B$ is finite, then $|A|\le |B|$. In particular, if $A \subseteq B$ then $|A|\le |B|$.
        \item If $g:B\rightarrow A$ is a surjection and $B$ is finite then $|A|\le |B|$.
        \item If $A$ is finite and $B$ is any set then $|A\cap B|\le |A|$ and $|A\setminus B|\le |A|$
    \end{itemize}
    \item \textbf{Thrm:} If $A$ and $B$ are finite sets, then
    \begin{itemize}
        \item $|A\times B| = |A| * |B|$
        \item $|A\cup B| = |A| + |B| - |A \cap B|$
    \end{itemize}
    \item \textbf{Lemma:} If $A \subseteq \bb{N}$ is finite and nonempty, then $A$ has a maximum element.
    \item \textbf{Thrm:} $\bb{N}$ is infinite.
    \item \textbf{Defn:} A set $A$ is 
    \begin{itemize}
        \item \emph{countably infinite} iff $|A| = |\bb{N}|$
        \item \emph{countable} (or \emph{listable} or \emph{denumerable}) iff $A$ is finite or countably infinite
        \item \emph{uncountable} iff $A$ is not countable
        \item (Note: every set is either finite, countably infinite, or uncountable)
    \end{itemize}
    \item \textbf{Thrm:} 
    \begin{itemize}
        \item For all $n\in\bb{Z}^+, |\bb{N}^n|=|\bb{N}|$
        \item If $n\in\bb{Z}^+$ and $X_1,X_2,\dots,X_n$ are nonempty countable sets, then
        \begin{itemize}
            \item $\prod_{i=1}^{n}X_1$ is countable
            \item If at least one $X_i$ is infinite, then $\prod_{i=1}^{n}X_i$ is countably infinite.
        \end{itemize}
    \end{itemize}
    \item Let $X$ be a nonempty set. Then the following are equivalent.
    \begin{itemize}
        \item $X$ is countable
        \item There exists an injection $f: X\rightarrow \bb{N}$
        \item There exists a surjection $g: \bb{N} \rightarrow X$
    \end{itemize}
    \item \textbf{Thrm:} $|\bb{Q}| = |\bb{N}|$
    \item \textbf{Thrm} (Countable Union of Countable Sets is Countable): If $\{A_n \mid n\in\bb{N}\}$ is a family of countable sets, then $\bigcup_{n\in\bb{N}}A_n$ is countable.
    \item $\bb{R}$ is uncountable
    \item For any set $S$, $|S| \ne |\mathcal{P{(S)}}|$ (Cantor's theorem)
    \item $|A| \le |B|$ iff there exists an injection $f:A\rightarrow B$ ($\ge$ vice versa)
    \begin{itemize}
        \item this is transitive
    \end{itemize}
    \item \textbf{Thrm} (Schr\"oder-Bernstein Thrm): If there exist injections $f:A\rightarrow B$ and $g:B\rightarrow A$, then there exists an injection $h:A\rightarrow B$.
\end{itemize}

\addsection{Number Theory}
\begin{itemize}
    \item a \emph{unit} is the natural number $n=1$
    \item a natural $>$ 1 is \emph{prime} iff its only positive divisors are 1 and itself
    \item a natural $>$ 1 is \emph{composite} iff $n$ is not prime (ie $\exists a,b\in\bb{Z},(1<a\le b < n \rightarrow n=ab)$)
    \item \emph{coprime} iff the gcd of two integers is 1
    \item every integer $n\ge 2$ is either prime or the product of primes (induction)
    \item Let $m,n$ be nonzero integers. If $m|n$ then $|m|\le|n|$
    \begin{itemize}
        \item \textbf{Corollary:} If $m|n \land n|m$ then $m=n\lor m=-n$
        \item if $n\in\bb{Z}$ and $n\ne 0$, then $n$ only has a finite number of divisors, since $m|n\rightarrow |m|\le|n|$
    \end{itemize}
    \item if $n\in\bb{N}$ is composite, then $n$ has a prime factor $\le \sqrt{n}$
    \item for $a,b\in\bb{Z}$, not both 0, define the \emph{greatest common divisor} of $a$ and $b$, denoted $\gcd(a,b)$, as the largest $d\in\bb{Z}$ s.t. $d|a \land d|b$.
    \begin{itemize}
        \item ie $d=\gcd(a,b) \Leftrightarrow d|a \land d|b \land \forall c\in\bb{Z},((c|a \land c|b)\rightarrow c\le d)$
        \item $\gcd(a,b) = \gcd(|a|,|b|)$
        \item note that $\gcd(0,b) = |b|$
    \end{itemize}
    \item Let $a,b\in\bb{Z}$, not both 0, and $d=\gcd(a,b)$. Then $\frac{a}{d}$ and $\frac{b}{d}$ are coprime
    \item The Euclidean Algorithm
    \begin{itemize}
        \item EA Lemma: Let $a,b\in\bb{Z}$ with $b\ne 0$. Then $\gcd(a,b) = \gcd(b,a-bk)$ for any integer $k$.
        \item Let $a,b\in\bb{Z}$ with $a>b>0$. To find $\gcd(a,b)$: (transcribe formal)
        \begin{align*}
            &\text{Consider 148 and 40} \\
            148 &= 40*3 + 28 \\
            40 &= 28*1 + 12 \\
            28 &= 12*2 + 4 \\
            12 &= 4*3 + 0
        \end{align*}
        Last nonzero remainder (4) is the gcd
    \end{itemize}
    \item Bezout's Lemma. Let $a,b\in\bb{Z}$, not both 0. Then
    \begin{itemize}
        \item $\exists x,y\in \bb{Z}, (ax+by=\gcd(a,b))$
        \item If $x,y\in\bb{Z}$ s.t. $ax+by>0$ then $ax+by\ge \gcd(a,b)$
        \item ie $\gcd(a,b)$ is the smallest positive integer of the form $ax+by$
        \item Corollaries (Let $a,b\in\bb{Z}$, not both 0):
        \begin{itemize}
            \item If $d=\gcd(a,b)$ and $t\in\bb{Z}$ then $(t|a \land t|b)\Leftrightarrow t|d$
            \begin{itemize}
                \item (any common divisor of $a$ and $b$ divides $\gcd(a,b)$ and vice versa)
            \end{itemize}
            \item $\forall c\in\bb{Z}, (\exists x,y\in\bb{Z}, (ax+by=c) \Leftrightarrow \gcd(a,b) | c)$
            \item $a$ and $b$ are coprime iff $\exists x,y\in\bb{Z}, (ax+by = 1)$
            \item $\forall m\in\bb{Z}^+,(\gcd(ma,mb)=m*\gcd(a,b))$
        \end{itemize}
    \end{itemize}
    \item Linear Diophantine Equations
    \begin{itemize}
        \item Diophantine equations are polynomial equations, typically in several variables, in which integer solutions are desired.
        \item For $a,b,c\in\bb{Z}$, a Diophantine equation of the form $ax+by=c$ is called a linear Diophantine equation in 2 variables
        \item We know that $\exists x,y\in\bb{Z}, (ax+by=c)$ iff $\gcd(a,b)\mid c$. We can find such $x,y$ with reverse Euclidean alg. \emph{(transcribe)}
        \item how do we find all such solutions? 
        \begin{itemize}
            \item \textbf{Thrm:} If $d=\gcd(a,b)$ and $d\mid c$ then there are infinitely many integer solutions to $ax+by=c$. Moreover, if $(x_0,y_0)$ is one such solution, then the set of all solutions is $$\{(x_0+\frac{bk}{d},y_0-\frac{ak}{d})\mid k\in\bb{Z}\}$$
        \end{itemize}
    \end{itemize}
    \item The Least Common Multiple
    \begin{itemize}
        \item For $a,b\in\bb{Z}$, not both 0, we definte the \emph{least common multiple of $a$ and $b$}, denoted $\lcm[a,b]$, as the smallest $c\in\bb{Z}^+$ s.t. $a\mid c \land b\mid c$
        \item Let $a,b\in\bb{Z}$, both nonzero. Then $\forall n\in\bb{Z},(\lcm[a,b]\mid n \Leftrightarrow a\mid n \land b\mid n)$
        \begin{itemize}
            \item ie Any common multiple is divisible by the least common multiple
            \item Corollaries: Let $a,b\in\bb{Z}$, not both 0. Then,
            \begin{itemize}
                \item $a\mid b \Leftrightarrow \lcm[a,b] = |b|$
                \item $\forall m\in\bb{Z}^+, \lcm[ma,mb] = m*\lcm[a,b]$
            \end{itemize}
        \end{itemize}
        \item (GCD-LCM Theorem). $\forall a,b\in\bb{Z}^+,\gcd(a,b)*\lcm[a,b] = ab$
    \end{itemize}
    \item Prime Factorizations
    \begin{itemize}
        \item Euclid's Lemma. Let $a,b,c\in\bb{Z}$ with $\gcd(a,b)=1$. If $a\mid bc$ then $a\mid c$.
        \begin{itemize}
            \item Corollary (also Euclid's Lemma). Let $p,a,b\in\bb{Z}$ with $p$ prime. If $p\mid ab$ then $p\mid a$ or $p\mid b$.
        \end{itemize}
        \item Fundamental Theorem of Arithmetic. Every natural number $n>1$ can be written uniquely as a product of prime numbers. (Unique up to reordering of the factors).
        \begin{itemize}
            \item ie If $n=p_1p_2\dots p_r = q_1q_2\dots q_s$ for primes $p_i,q_j$ s.t. $p_1\le p_2\le \dots \le p_r$ and $q_1\le q_2\le \dots \le q_s$ then $r=s$ and $p_i=q_i$ for all $i$.
            \item Corollary: exist infinitely many primes.
        \end{itemize}
    \end{itemize}
    \item Divisors
    \begin{itemize}
        \item Let $\bb{P}$ be the set of prime numbers. For each $p\in \bb{P}$ and $n\in\bb{Z}^+$, define $v_p(n) = \max\{a\in\bb{Z} : p^a\mid n\}$
        \item note: If $a\mid n$ then $\forall p\in\bb{P},v_p(a)\le v_p(n)$
        \item Suppose $n$ has prime factorization ah fuck it
    \end{itemize}
\end{itemize}


%%%%%%%%%%%%%%%%%%%%%%%%%%%%
%% End of document body   %%
%%%%%%%%%%%%%%%%%%%%%%%%%%%%
\end{document}
